\section{Replication}

\subsection{File ownership}
To get a replicaiton factor of tree we use the node that was determined by using the legacy consistent hashing algorithm plus the two nodes that succeeded that node in the ring. The implementation of these new consistent hashing algorithms can be seen at \textbf{src/utils/algorithms/NeighbourhoodAlgorithms} and are called \textbf{findReplicationNodes} and \textbf{findReplicationHeirs} 

\subsection{Implications at the storage services}
\subsubsection{Retrieving a value}
Now the node that receives the get request gets a list of all the nodes that must have the file that the client requested. Then the node that the client contacted will establish a socket with each node of that list until a node that contains the value is found. This flow of execution can be observed at \textbf{src/store/handlers/store/GetRequestHandler execute} method

\subsubsection{Storing a value}
When the client asks the node to store a value. The node determines, by using the plain consistent hashing algorithm if the key belongs to that node. If so it stores the value and replicates that value to the next two nodes. If the file dosen't belong to that node the node creates a connection with the node that was supposed to store the file and that node handles the file storing like described above if the connection fails it moves to the next node in the line and tries to warn all the other nodes that that node is down. The replication flag here is used to say if a node that received a PUT must try to replicate that value to the other nodes. Usually this value is true for requests sent by the client and false otherwise. This flow of execution can be observed at \textbf{src/store/handlers/store/PutRequestHandler execute} method

\subsubsection{Deleting a pair}
When the client wants to delete a pair the contacted node determines, by using the plain consistent hashing algorithm if the request must be handled by that node. If so, it replaces the pair by a tombstone that is represented by a file whose name is the former key contatenated with \texttt{"\_DEL"} the content of that file is the time in millisecond of when the pair was deleted. If the node dosen't handle that request it sends the request to the correct node that then replicates the request for all the nodes that handle that pair. This flow of execution can be observed at \textbf{src/store/handlers/store/DeleteRequestHandler execute} method.

\subsection{Periodic Tasks}
We created periodic tasks that ran in threads to both mantain the replication factor and to delete old tombstones. The subclass that is the parent of every task is located at \textbf{src/service/periodic/PeriodicActor}
\subsubsection{Maintaining the replication factor}
To maintain the replication factor we run a method that in all nodes iterate over the pairs that it possesses, computes the nodes that must hold that pair, asks the nodes if they have that pair using a SEEK request that is similar to a GET request but only sends a boolean response that says if the node has a given file or not, and if the node dosen't have that pair it sends it. This task runs every 15 seconds. The code that runs every time can be found at \textbf{src/service/periodic/CheckReplicationFactor::run}

\subsubsection{Deleting old tombstones}
A user might have deleted a file by mistake. In order to prevent backlisting files forever this task deletes all tombstones that have more than 5 seconds. This task runs every 5 seconds. \textbf{src/service/periodic/PeriodicDeleteTombstones::run}